\documentclass[spanish,xcolor=table]{beamer}
\usepackage[spanish]{babel}
\usepackage{graphicx}
\selectlanguage{spanish}
\usepackage[utf8]{inputenc}
%-------------------Package----------------------------
\RequirePackage{natbib}
\usepackage{booktabs} 
\usepackage{tikz}
\usepackage{pgfplots}
\usepackage{multirow}
\usepackage{multicol}
\usepackage{amsmath}
\usepackage{hhline}
\usepgfplotslibrary{groupplots} 
\usepackage{adjustbox} 
\usepackage{relsize}
\usepackage[nointegrals]{wasysym}
\usepackage{caption}
\usepackage{bigints}
\usepackage{xcolor}
%---Librerias de Tikz -----------------------------
\usetikzlibrary{arrows,calc}
\usetikzlibrary{intersections}
\usepgfplotslibrary{fillbetween}
\usetikzlibrary{patterns}
\usetikzlibrary{plotmarks}
\usetikzlibrary{calc}
\usetikzlibrary{matrix}
\usetikzlibrary{positioning}
\usetikzlibrary{decorations.pathreplacing}
%-------------------------------------------------

%-----------Tikz Layers---------------------------------
\pgfdeclarelayer{ft}
\pgfdeclarelayer{bg}
\pgfsetlayers{bg,main,ft}
%---------------------------------------------------------



%-----north west lines pattern density ajusted------------------
\makeatletter 
\pgfdeclarepatternformonly[\LineSpace,\tikz@pattern@color]{MNWL}{\pgfqpoint{-1pt}{-1pt}}{\pgfqpoint{\LineSpace}{\LineSpace}}{\pgfqpoint{\LineSpace}{\LineSpace}}%
{
    \pgfsetcolor{\tikz@pattern@color} 
    \pgfsetlinewidth{0.4pt}
    \pgfpathmoveto{\pgfqpoint{0pt}{0pt}}
    \pgfpathlineto{\pgfqpoint{\LineSpace + 0.1pt}{\LineSpace + 0.1pt}}
    \pgfusepath{stroke}
}
\makeatother 
\newdimen\LineSpace
\tikzset{
    line space/.code={\LineSpace=#1},
    line space=6pt %set density of lines
}

%----------------Beamer Themes--------------------

%\usetheme{default}
%\usetheme{AnnArbor}
%\usetheme{Antibes}
%\usetheme{Bergen}
%\usetheme{Berkeley}
%\usetheme{Berlin}
%\usetheme{Boadilla}
\usetheme{CambridgeUS}
%\usetheme{Copenhagen}
%\usetheme{Darmstadt}
%\usetheme{Dresden}
%\usetheme{Frankfurt}
%\usetheme{Goettingen}
%\usetheme{Hannover}
%\usetheme{Ilmenau}
%\usetheme{JuanLesPins}
%\usetheme{Luebeck}
%\usetheme{Madrid}
%\usetheme{Malmoe}
%\usetheme{Marburg}
%\usetheme{Montpellier}
%\usetheme{PaloAlto}
%\usetheme{Pittsburgh}
%\usetheme{Rochester}
%\usetheme{Singapore}
%\usetheme{Szeged}
%\usetheme{Warsaw}

%--------------------Beamer Font & Colors----------

%\usecolortheme{albatross}
%\usecolortheme{beaver}
%\usecolortheme{beetle}
%\usecolortheme{crane}
\usecolortheme{dolphin}
%\usecolortheme{dove}
%\usecolortheme{fly}
%\usecolortheme{lily}
%\usecolortheme{orchid}
%\usecolortheme{rose}
%\usecolortheme{seagull}
%\usecolortheme{seahorse}
%\usecolortheme{whale}
%\usecolortheme{wolverine}

%-------------------Beamer Options --------------------
%\setbeamertemplate{footline} % To remove the footer line in all slides uncomment this line

%\setbeamertemplate{footline}[page number] % To replace the footer line in all slides with a simple slide count uncomment this line

\setbeamertemplate{navigation symbols}{} % To add the navigation symbols from the bottom of all slides comment this line


%-------------------Package----------------------------
\RequirePackage{natbib}
\usepackage{graphicx} 
\usepackage{booktabs} % use of \toprule, \midrule and \bottomrule in tables
\usepackage{tikz}
\usepackage{pgfplots}
\usepackage{multirow}
\usepackage{multicol}
\usepackage{amsmath}
\usepackage{hhline}

%---Librerias de Tikz -----------------------------
\usetikzlibrary{arrows,calc}
\usetikzlibrary{intersections}
\usetikzlibrary{pgfplots.fillbetween}
\usetikzlibrary{patterns}
\usetikzlibrary{plotmarks}
\usetikzlibrary{calc}
\usetikzlibrary{matrix}
\usetikzlibrary{positioning}
\usepackage{relsize}
\usepackage[nointegrals]{wasysym}
\usepackage{caption}
%\captionsetup[table]{name=Tabla} % Cambia caption table
\usepackage{bigints}
\usepackage{xcolor}
%---------------------------------------------------------------------------------

%-----------Tikz Layers---------------------------------
\pgfdeclarelayer{ft}
\pgfdeclarelayer{bg}
\pgfsetlayers{bg,main,ft}
%---------------------------------------------------------
%----------------------------------------------------------------------------------------
%	           Slide Inicial
%----------------------------------------------------------------------------------------
 
%-------------------------------
%      UNCOMMENT AS NEEDED     -
%-------------------------------
%Secciones
 
\title{Ayudantía 18 Métodos Matemáticos II}
\author{Martín Armas Ll.}
\institute[FEN]{Universidad de Chile}
  % institution for the title page}
\medskip
%-----------------------------------------------------------------


\date{}

%-------------------Figure Settings-------------------
\pgfplotsset{ % Here we specify options for all figures in the document
  compat=1.8, % Which version of pgfplots do we want to use?
  legend style = {font=\small\sffamily}, % Legends in a sans-serif fonti
  label style = {font=\small\sffamily} % Labels in a sans-serif font
}

%-----Color definitions-------------
\colorlet{ColorG}{black!60!green}
\colorlet{ColorR}{black!60!red}
\colorlet{ColorB}{black!60!blue}
\colorlet{ColorY}{black!40!yellow}
%-----------------------------------

%%%%%%%%%%%%%%%%%%%%%%%%%%%%%
%%%%%%%%%%%%%%%%%%%%%%%%%%%%%


\begin{document}


\begin{frame}

%\begin{figure}[t!]
%\includegraphics[scale=0.1]{Figuras/Logo.jpg}
%\end{figure}
\titlepage % Print the title page as the first slide
\end{frame}





%----------------------------------------------------------------------------------------
%	PRESENTATION SLIDES
%----------------------------------------------------------------------------------------
\begingroup
\setbeamertemplate{page number in head/foot}{}
\begin{frame}[noframenumbering]
\frametitle{Contenidos} 
\tableofcontents
\end{frame}
\endgroup
%%%%%%%%%%%%%%%%%%%%%%%%%%%%%%%%%

\section{Formas cuadráticas y signos de una matriz}

\begin{frame}{Pregunta 1}

Evaluadas en $X\in \mathbb{R}^2$, encuentre las formas cuadráticas asociadas a las siguientes matrices:

$$A=\begin{bmatrix}1&2\\2 & \alpha \end{bmatrix}\;\;\;
B=\begin{bmatrix}0&1\\1 & 1 \end{bmatrix}$$

\end{frame}


\begin{frame}{Respuesta 1}



\end{frame}


\begin{frame}{Respuesta 1}

Sea $X\in \mathbb{R}^2$ un vector que viene dado por $X=\begin{bmatrix} x_1 \\ x_2\end{bmatrix}$, tenemos que:

\begin{equation*}
    \begin{split}
        Q_A(X)&=X^tAX\\
        &=\begin{bmatrix} x_1 & x_2\end{bmatrix} \begin{bmatrix}1&2\\2 & \alpha \end{bmatrix}  \begin{bmatrix} x_1 \\ x_2\end{bmatrix}\\
        &=\begin{bmatrix} x_1+2x_2 & 2x_1+\alpha x_2\end{bmatrix}\begin{bmatrix} x_1 \\ x_2\end{bmatrix}\\
        &=x^2_1+\alpha x^2_2+4x_1x_2
    \end{split}
\end{equation*}

\end{frame}


\begin{frame}{Respuesta 1}

Sea $X\in \mathbb{R}^2$ un vector que viene dado por $X=\begin{bmatrix} x_1 \\ x_2\end{bmatrix}$, tenemos que:

\begin{equation*}
    \begin{split}
        Q_B(X)&=X^tBX\\
        &=\begin{bmatrix} x_1 & x_2\end{bmatrix} \begin{bmatrix}0&1\\1 & 1 \end{bmatrix}  \begin{bmatrix} x_1 \\ x_2\end{bmatrix}\\
        &=\begin{bmatrix} x_2 & x_1+x_2\end{bmatrix}\begin{bmatrix} x_1 \\ x_2\end{bmatrix}\\
        &=x^2_2+2x_1x_2
    \end{split}
\end{equation*}

\end{frame}


\begin{frame}{Pregunta 2}

Encuentre la matriz simétrica que está asociada a la siguiente forma cuadrática:

$$Q_A(X)=x^2_1-4x^2_2+6x_1x_2$$

\end{frame}


\begin{frame}{Respuesta 2}



\end{frame}


\begin{frame}{Respuesta 2}

Tomemos la matriz $A$ cuyos componentes son $3$ incógnitas:
$$A=\begin{bmatrix}a&b\\b & c \end{bmatrix}$$
Sabemos que la forma cuadratica de $A$ viene dada por:


\begin{equation*}
    \begin{split}
        Q_A(X)&=X^tAX\\
        &=\begin{bmatrix} x_1 & x_2\end{bmatrix} \begin{bmatrix}a&b\\b & c \end{bmatrix}  \begin{bmatrix} x_1 \\ x_2\end{bmatrix}\\
        &=\begin{bmatrix} ax_1+bx_2 & bx_1+cx_2\end{bmatrix}\begin{bmatrix} x_1 \\ x_2\end{bmatrix}\\
        &=ax^2_1+cx^2_2+2bx_1x_2
    \end{split}
\end{equation*}

\end{frame}


\begin{frame}{Respuesta 2}

Igualando con la forma cuadratica dada por el enunciado, observamos que se debe cumplir $a=1$, $c=-4$ y $b=3$. Cualquier matriz que cumpla con estas condiciones, será una matriz asociada a la forma cuadrática $Q_A(X)$. Por ejemplo, una de estas matrices es:

$$A=\begin{bmatrix}1&3\\3 & -4 \end{bmatrix}$$

\end{frame}


\begin{frame}{Pregunta 3}

Dadas constantes $a,b,c$, definamos la matriz:

$$A=\begin{bmatrix}a&0&0\\0&b&0\\0&0&c \end{bmatrix}$$

Dado $X\in \mathbb{R}^3$, muestre entonces que la forma cuadrática asociada a la matriz $A$ es:

$$Q_A(X)=ax^2_1+bx^2_2+cx^2_3$$

\end{frame}


\begin{frame}{Respuesta 3}



\end{frame}


\begin{frame}{Respuesta 3}

Sea $X\in \mathbb{R}^3$ un vector que viene dado por $X=\begin{bmatrix} x_1 \\ x_2\\ x_3 \end{bmatrix}$, tenemos que:

\begin{equation*}
    \begin{split}
        Q_A(X)&=X^tAX\\
        &=\begin{bmatrix} x_1 & x_2 & x_3\end{bmatrix} \begin{bmatrix}a&0&0\\0 & b& 0 \\ 0&0&c \end{bmatrix}  \begin{bmatrix} x_1 \\ x_2 \\x_3\end{bmatrix}\\
        &=\begin{bmatrix} ax_1&bx_2 & cx_3\end{bmatrix}\begin{bmatrix} x_1 \\ x_2\\x_3\end{bmatrix}\\
        &=ax^2_1+bx^2_2+cx^2_3
    \end{split}
\end{equation*}

\end{frame}


\begin{frame}{Pregunta 4}

Determine el signo de las siguientes matrices:

$$A=\begin{bmatrix}2&0\\0 & 0 \end{bmatrix}\;\;\;
B=\begin{bmatrix}1&0\\0 & -1 \end{bmatrix}$$

\end{frame}


\begin{frame}{Respuesta 4}



\end{frame}


\begin{frame}{Respuesta 4}

Partimos por la matriz A. Calculamos su forma cuadrática:

\begin{equation*}
    \begin{split}
        Q_A(X)&=X^tAX\\
        &=\begin{bmatrix} x_1 & x_2\end{bmatrix} \begin{bmatrix}2&0\\0 & 0 \end{bmatrix}  \begin{bmatrix} x_1 \\ x_2\end{bmatrix}\\
        &=\begin{bmatrix} 2x_1 & 0\end{bmatrix}\begin{bmatrix} x_1 \\ x_2\end{bmatrix}\\
        &=2x^2_1
    \end{split}
\end{equation*}

Es fácil ver que para todo valor de $x_1\in \mathbb{R}$ se cumple $Q_A(X)>0$, por lo que la matriz $A$ es definida positiva.

\end{frame}


\begin{frame}{Respuesta 4}

Ahora, calculamos la forma cuadrática de la matriz B:

\begin{equation*}
    \begin{split}
        Q_B(X)&=X^tBX\\
        &=\begin{bmatrix} x_1 & x_2\end{bmatrix} \begin{bmatrix}1&0\\0 & -1 \end{bmatrix}  \begin{bmatrix} x_1 \\ x_2\end{bmatrix}\\
        &=\begin{bmatrix} x_1 & -x_2\end{bmatrix}\begin{bmatrix} x_1 \\ x_2\end{bmatrix}\\
        &=x^2_1-x^2_2
    \end{split}
\end{equation*}

Podemos ver que si $|x_1|>|x_2|$ tenemos $Q_B>0$, pero si $|x_1|<|x_2|$ tenemos $Q_B<0$. Luego, al depender de los valores que tomen $x_1,x_2$, la matriz no tiene signo.

\end{frame}


\begin{frame}{Pregunta 5}

Explique por qué si $A\in \mathbb{R}^{n\times n}$ y $B\in \mathbb{R}^{n\times n}$ son matrices definidas positivas, entonces $A+B$ es una matriz definida positiva.

\end{frame}


\begin{frame}{Respuesta 5}



\end{frame}


\begin{frame}{Respuesta 5}

El hecho de que las matrices $A$ y $B$ sean matrices definidas positivas, nos informan que, para todo vector $X \in \mathbb{R}^n$ se tiene:

$$ (X^tAX&>0) \wedge (X^tBX&>0) $$

Sabemos que la forma cuadrática de la matriz $A+B$ viene dada por:

\begin{equation*}
    \begin{split}
        Q_{A+B}(X)&=X^t(A+B)X\\
        &=(X^tA+X^tB)X\\
        &=X^tAX+X^tBX
    \end{split}
\end{equation*}
Dado que $Q_{A+B}(X)$ se conforma por la suma de dos numeros positivos, tenemos que $Q_{A+B}(X)>0$ para todo $X \neq 0_n$. Concluimos que la matriz $A+B$ es definida positiva.

\end{frame}


\begin{frame}{Pregunta 6}

Explique por qué si $A\in \mathbb{R}^{n\times n}$ es una matriz definida positiva, entonces $-A$ es una matriz definida negativa.

\end{frame}


\begin{frame}{Respuesta 6}



\end{frame}


\begin{frame}{Respuesta 6}

El hecho de que $A$ es una matriz definida positiva, nos informan que, para todo vector $X \in \mathbb{R}^n$ se tiene que $ X^tAX&>0 $

Sabemos que la forma cuadrática de la matriz $-A$ viene dada por:

\begin{equation*}
    \begin{split}
        Q_{-A}(X)&=X^t(-A)X\\
        &=-X^tAX\\
        &=-(X^tAX)
    \end{split}
\end{equation*}
Dado que $Q_{-A}$ se conforma por la multiplicación de un número negativo con uno positivo, tenemos que $Q_{-A}<0$ para todo $X \neq 0_n$. Concluimos que la matriz $-A$ es definida negativa.

\end{frame}

\end{document}